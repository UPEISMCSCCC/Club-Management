\documentclass[11pt]{article}

\usepackage{geometry}
\geometry{a4paper,
		  lmargin=2.2cm,
		  rmargin=2.2cm,
		  tmargin=2.4cm,
		  bmargin=2.8cm}

% font setup
\usepackage[T1]{fontenc}
\usepackage{lmodern}

\begin{document}

% TITLE
{\fontsize{20}{24}\center\bfseries Competitive Coding Club Constitution \par}

\section{Name}

The club's name is the \textit{Competitive Coding Club}.

\section{Objectives}

\textit{What does the group represent?}

\noindent
The group represents those interested in competitive programming, and those who partake in various programming competitions.

\vspace{2mm}
\noindent
\textit{What is the purpose of the group?}

\noindent
The purpose is to provide resources for competitive programmers at UPEI and to encourage more students to partake in programming competitions.

\vspace{2mm}
\noindent
\textit{What would the group like to achieve?}

\noindent
The goal is to increase participation and improve results at various competitions that UPEI partakes in, such as the annually-run International Collegiate Programming Contest (ICPC).

\section{Membership}

Membership is simple and is done by informing any of the executive members of the club that one would wish to join. This practice is kept relatively informal to keep the process simple.

\vspace{2mm}
\noindent
There are no restrictions on who is eligible to join and anyone is free to do so, and there are no fees associated with joining.

\section{Activities}

There are two main activities:

\vspace{2mm}
\noindent
In preparation for ICPC, where teams of three or less must be prepared in order to compete in the regional qualifiers, a practice competition is to be run. This not only provides practice with competitive programming problems, but also familiarizes people with the format and with working within a team. Depending on interest and number of teams, this may or may not be used as a preliminary qualifier for deciding which teams to send forward and compete with.

\vspace{2mm}
\noindent
A completely self-contained competition for students to compete individually and practice their skills on their own. A variety of competitive programming problems of various difficulty is to be given, and contestants are to attempt to solve as many as they can within a time frame.

\section{Executive}

The executive council is formed of the following members:

\begin{itemize}
  \item President
  \item Vice President
  \item Treasurer
  \item Record Keeper
\end{itemize}

\subsection{Executive Duties}

\subsection{Appointments}

\subsection{Vacancies}

While a member of the executive is unavailable or the position is unfilled, the duties fall to other members of the executive to perform in a manner coordinated by the other executives.

If the position remains unfilled, other members of the executive can find willing candidates who would like to fill that position.

\subsection{Removals}

A member on the executive is able to remove themselves from their position at any time, though it is recommended to consult other members.

There exists a process for removing a member of the executive, if they are found unable to perform their duties. A simple majority vote among the executive (excluding the member being voted on) can be performed to determine whether that member is to be forcibly removed from their position.

\section{Meetings}

\section{Finance}

\section{Amendments}

\end{document}
