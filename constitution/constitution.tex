\documentclass[11pt]{article}

\usepackage{geometry}
\geometry{a4paper,
		  lmargin=2.2cm,
		  rmargin=2.2cm,
		  tmargin=2.4cm,
		  bmargin=2.8cm}

% font setup
\usepackage[T1]{fontenc}
\usepackage{lmodern}

\begin{document}

% TITLE
{\fontsize{20}{24}\center\bfseries Competitive Coding Club Constitution \par}

\section{Name}

The club's name is the \textit{Competitive Coding Club}.

\section{Objectives}

\textit{What does the group represent?}

\noindent
The group represents those interested in competitive programming, and those who partake in various programming competitions.

\vspace{2mm}
\noindent
\textit{What is the purpose of the group?}

\noindent
The purpose is to provide resources for competitive programmers at UPEI and to encourage more students to partake in programming competitions.

\vspace{2mm}
\noindent
\textit{What would the group like to achieve?}

\noindent
The goal is to increase participation and improve results at various competitions that UPEI partakes in, such as the annually-run International Collegiate Programming Contest (ICPC).

\section{Membership}

Membership is simple and is done by informing any of the executive members of the club that one would wish to join. This practice is kept relatively informal to keep the process simple.

\vspace{2mm}
\noindent
There are no restrictions on who is eligible to join and anyone is free to do so, and there are no fees associated with joining.

\section{Activities}

There are two main activities:

\vspace{2mm}
\noindent
In preparation for ICPC, where teams of three or less must be prepared in order to compete in the regional qualifiers, a practice competition is to be run. This not only provides practice with competitive programming problems, but also familiarizes people with the format and with working within a team. Depending on interest and number of teams, this may or may not be used as a preliminary qualifier for deciding which teams to send forward and compete with.

\vspace{2mm}
\noindent
A completely self-contained competition for students to compete individually and practice their skills on their own. A variety of competitive programming problems of various difficulty is to be given, and contestants are to attempt to solve as many as they can within a time frame.

\section{Executive}

The executive council is formed of the following members:

\begin{itemize}
  \item President
  \item Vice President
  \item Treasurer
  \item Record Keeper
\end{itemize}

\subsection{Executive Duties}

All members of the executive have decision making powers, as decision making is generally done in the form of majority votes or mutual agreement among the executive.

\vspace{2mm}
\noindent
The President's duty is to maintain contact with various sources outside of the club, and to ensure that the club is ratified and running functionally. This can include functions such as organizing meetings with the rest of the executive.

\vspace{2mm}
\noindent
The Vice President's duty is to aid in the president's duties, and to temporarily take over said duties whenever the president is unable to.

\vspace{2mm}
\noindent
The Treasurer's duty is to manage funding -- to approve or deny various expenditures, and to determine the amount of funding needed to perform various activities.

\vspace{2mm}
\noindent
The Record Keeper's duty is to ensure that all meetings are documented and recorded in text, to create a historical record of decisions made.

\subsection{Appointments}

At the start of every year, any potential president is to be voted on by the current members of the club. This process consists of determining which members would like to run for president, then (if there are multiple candidates) contacting all members and informing them to vote anonymously for their president. The president then chooses various willing members to fill various remaining members of the executive.

\subsection{Vacancies}

While a member of the executive is unavailable or the position is unfilled, the duties fall to other members of the executive to perform in a manner coordinated by the other executives.

\vspace{2mm}
\noindent
If the position remains unfilled, other members of the executive can find willing candidates who would like to fill that position. If there are multiple such candidates, the executive consults and votes on which candidate they would like to see fill the position.

\vspace{2mm}
\noindent
Alternatively, filling a position may be done by changing the responsibilities of the current executive, after agreement with the rest of the remaining executive. For instance, if one member were to resign, another member would be able to take on the role of the first member. It would then be the second member's old position that's unfilled and needs to be filled.

\subsection{Removals}

A member on the executive is able to resign themselves from their position at any time, though it is recommended to consult other members.

\vspace{2mm}
\noindent
There exists a process for removing a member of the executive, if they are found unable to perform their duties. A simple majority vote among the executive (excluding the member being voted on) can be performed to determine whether that member is to be forcibly removed from their position.

\section{Meetings}

Meetings are to be scheduled and run by the executive. The president, or whoever is performing the duties of the president, is to organize with the other members of the executive when and how to meet.

\vspace{2mm}
\noindent
In order for decisions to be made during a meeting, at least 3 members of the executive must be present.

\section{Finance}

\section{Amendments}

If a change or addition to this constitution has a clear majority of the executive in approval (3/4 or 4/4 in support of the amendment) it is to be discussed and implemented. This constitution is to be updated with any amendments before it is put in practice, and must be done so before the start of the next semester or end of academic year.

\end{document}
